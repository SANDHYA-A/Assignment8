\documentclass[journal,12pt,twocolumn]{IEEEtran}
\IEEEoverridecommandlockouts
\usepackage{cite}
\usepackage{amsmath,amssymb,amsfonts,bm}
\usepackage{mathtools}
\usepackage{tkz-euclide} 
\usepackage{tikz}
\usetikzlibrary{calc,math}
 \usepackage{caption}
\usepackage{listings}
\usepackage{gensymb}
\let\vec\mathbf
\numberwithin{equation}{subsection}

\newcommand{\myvec}[1]{\ensuremath{\begin{pmatrix}#1\end{pmatrix}}}
\newcommand{\norm}[1]{\left\lVert#1\right\rVert}
\newcommand{\mydet}[1]{\ensuremath{\begin{vmatrix}#1\end{vmatrix}}}

\renewcommand\thesection{\arabic{section}}
\renewcommand\thesubsection{\thesection.\arabic{subsection}}
\renewcommand\thesubsubsection{\thesubsection.\arabic{subsubsection}}

\renewcommand\thesectiondis{\arabic{section}}
\renewcommand\thesubsectiondis{\thesectiondis.\arabic{subsection}}
\renewcommand\thesubsubsectiondis{\thesubsectiondis.\arabic{subsubsection}}
%\renewcommand{\theequation}{\theenumi}
%\numberwithin{equation}{enumi}

\providecommand{\mbf}{\mathbf}
\providecommand{\pr}[1]{\ensuremath{\Pr\left(#1\right)}}
\providecommand{\qfunc}[1]{\ensuremath{Q\left(#1\right)}}
\providecommand{\sbrak}[1]{\ensuremath{{}\left[#1\right]}}
\providecommand{\lsbrak}[1]{\ensuremath{{}\left[#1\right.}}
\providecommand{\rsbrak}[1]{\ensuremath{{}\left.#1\right]}}
\providecommand{\brak}[1]{\ensuremath{\left(#1\right)}}
\providecommand{\lbrak}[1]{\ensuremath{\left(#1\right.}}
\providecommand{\rbrak}[1]{\ensuremath{\left.#1\right)}}
\providecommand{\cbrak}[1]{\ensuremath{\left\{#1\right\}}}
\providecommand{\lcbrak}[1]{\ensuremath{\left\{#1\right.}}
\providecommand{\rcbrak}[1]{\ensuremath{\left.#1\right\}}}

\lstset{
frame=single, 
breaklines=true,
columns=fullflexible
}

\begin{document}

\title{Matrix Theory EE5609 - Assignment 8\\
}

\author{\IEEEauthorblockN{Sandhya Addetla}\\
\IEEEauthorblockA{PhD Artificial Inteligence Department} \\
AI20RESCH14001\\
 }

\maketitle
\begin{abstract}
Perform QR decomposition of a $3 \times 3$ matrix.
\end{abstract}
Download  python code from 
\begin{lstlisting}
https://github.com/SANDHYA-A/Assignment8
\end{lstlisting}
\section{Problem}
Perform QR decomposition of the matrix
\begin{align}
   \vec{V} = \myvec{9&0&-6\\0&-4&0\\-6&0&1}  \label{1.1}
\end{align}
\section{Solution}
Any matrix A can be converted in the form 
\begin{align}
    \vec{A} = \vec{Q}\vec{R}\label{2.1}
\end{align}
Here  $\vec{Q}$ is an orthogonal matrix and $\vec{R}$ is an upper triangular matrix. This is known as QR decomposition.

For the given matrix at \ref{1.1}, column vectors are,
\begin{align}
    \vec{a}=\myvec{9\\0\\-6} \quad \vec{b}=\myvec{0\\-4\\0} \quad \vec{c}=\myvec{-6\\0\\1}\label{2.2}
\end{align}
Equation  \ref{2.1} can be written in $\vec{Q}\vec{R}$ form as:
\begin{align}
    \vec{Q}\vec{R} = \myvec{\vec{q_1}&\vec{q_2}&\vec{q_3} }\myvec{r_1&r_2&r_3\\0&r_4&r_5\\0&0&r_6}  \label{2.3}
\end{align}
Where,
\begin{align}
r_1 = \norm{\vec{a}} = \sqrt{9^2+(-6)^2} = \sqrt{117} \label{2.4}\\
\vec{q_1} = \frac{\vec{a}}{r_1} = \myvec{\frac{9}{\sqrt{117}}\\0\\ \frac{-6}{\sqrt{117}}}
\end{align}
The value of $\vec{q_2}$ can be obtained as,
\begin{align}
r_2 = \vec{q_1}^T\vec{b}\\
= \myvec{\frac{9}{\sqrt{117}}&0& \frac{-6}{\sqrt{117}}}\myvec{0\\-4\\0}=0\\
r_4 = \norm{\vec{b}- r_2\vec{q_1}} = \sqrt{(-4)^2} = 4\\
\vec{q_2} = \frac{ \vec{b}- r_2\vec{q_1}}{ \norm{\vec{b}- r_2\vec{q_1}}}\\	
\vec{q_2} = \frac{1}{4}\myvec{0\\-4\\0} = \myvec{0\\-1\\0}
\end{align}
The value of $\vec{q_3}$ can be obtained as,
\begin{align}
r_3 = \vec{q_1}^T\vec{c}\\
= \myvec{\frac{9}{\sqrt{117}}&0& \frac{-6}{\sqrt{117}}}\myvec{-6\\0\\1} = \frac{-60}{\sqrt{117}}\\
r_5 = \vec{q_2}^T\vec{c}\\
=\myvec{0&-1&0}\myvec{-6\\0\\1} = 0\\
r_6 = \norm{\vec{c} - r_3  \vec{q_1} -r_5\vec{q_2}}\\
=\norm{\myvec{-6\\0\\1}+ \frac{60}{\sqrt{117}}\myvec{\frac{9}{\sqrt{117}}\\0\\ \frac{-6}{\sqrt{117}}}}\\
=\sqrt{\left(\frac{-18}{13}\right)^2 + \left(\frac{-27}{13}\right)^2 }=\frac{9}{\sqrt{13}}\\
\vec{q_3} = \frac{\vec{c} - r_3  \vec{q_1} -r_5\vec{q_2}}{ \norm{\vec{c} - r_3  \vec{q_1} -r_5\vec{q_2}}}\\
= \frac{\sqrt{13}}{9}\myvec{\frac{-18}{13}\\0\\ \frac{-27}{13}}= \myvec{\frac{-2}{\sqrt{13}}\\0\\ \frac{-3}{\sqrt{13}}} \label{2.20}
\end{align}
\\
\\
\\
\\
By substituting equation  \eqref{2.4} to  \eqref{2.20} in  \eqref{2.3},we obtain
the QR Decomposition of the given matrix as:
\begin{multline}
    \myvec{9&0&-6\\0&-4&0\\-6&0&1}  \\= \myvec{\frac{9}{\sqrt{117}}&0& \frac{-2}{\sqrt{13}}\\0&-1&0\\ \frac{-6}{\sqrt{117}}&0&\frac{-3}{\sqrt{13}}}\myvec{\sqrt{117} &0&\frac{-60}{\sqrt{117}}\\0&4&0 \\0&0&\frac{9}{\sqrt{13}}}
\end{multline}
\end{document}